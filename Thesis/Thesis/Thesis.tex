
\documentclass[oneside,a4paper]{book}
%\pagestyle{headings}

\input{preamble}
\graphicspath{{images/}}

% A B S T R A C T
% % % % % % % % % % % % % % % % % % % % % % % % % % % % % % % % % %
\chapter*{\centering Abstract}
\begin{quotation}
\noindent 
Abstract (max. 1 page)

Name of the Supervisor, Group, Institute, University, Supervisor

Name of the Assistant, Group, Institute, University, Assistant
\end{quotation}
\clearpage


% C O N T E N T S 
% % % % % % % % % % % % % % % % % % % % % % % % % % % % % % % % % % % % % % % %
\tableofcontents

%%%%%%%% Introduction %%%%%%%%
\chapter {Introduction}
\label {cha:introduction}
With recent successes such as AlphaGo defeating the reigning world champion and fast progress towards fully autonomous cars by Tesla as well as Weymo it is clear that Reinforcement Learning is the solution to many problems that could not even be tackled before. Many of these powerful implementations rely on equally powerful machines in order to train them, often requiring over hundred CPUs and GPUs for weeks on end. Inspired by these grand achievements and the technology behind them but lacking comparable resources I settled on a more achievable goal: Othello. This simple board game has accompanied me since the first semester when we were tasked to implement a search based Othello player. During my masters studies I suspected that a superior player could be created using machine learning techniques. This thesis documents the way to such a player and its key components.


%%%%%%%% Related Works %%%%%%%%
\chapter {Related Works}
\label {cha:Related Works}

\section {Othello}
\label{sec:Othello}

\begin{figure}[!b]
  \centering
	\includegraphics[scale=0.2]{OpeningPosition.png}
	\caption{The opening position}
	\label{fig:OpeningPosition}
\end{figure}

According to \cite{RULES} Othello is a game played on a  square board, usually made up of eight by eight tiles.
The opening position is shown in Figure \ref{fig:OpeningPosition}
Othello stones are black on one side and white on the other. Players take turns in placing one of these stones in their respective color on the board. Stones can only be placed in such a way that the new stone \textit{traps} one or more of the opponents stones inbetween itself and any other stone of the same color. All opposing stones that were trapped by placing the new stone are flipped and now belong to the player who trapped them. If a player cannot perform a legal move he simply passed and his opponent places the next stone. The game ends if both players pass successively or there are no free tiles left on the board. The player who controls more stones at this time wins the game.

\subsection {Search based Algorithms}
\label{sec:Search based Algorithms}
Othello is widely used to teach search based game theory algorithms, most notably the min-max algorithm and its optimization the alpha-beta pruning.


\subsection {Machine Learning based Algorithms}
\label{sec:Machine Learning based Algorithms}

\section {Machine learning}					% ?
\label{sec:Machine Learning}
\subsection {Convolutional Neural Networks}	% ?
\label{sec:Convolutional Neural Networks}

\section {Reinforcement Learning}
\label{sec:Reinforcement Learning}
\subsection {Monte Carlo Learning}
\label{sec:Monte Carlo Learning}
\subsection {Temporal Difference Learning }
\label{sec:Temporal difference Learning}
\subsection {Monte Carlo Tree Search}
\label{sec:Monte Carlo Tree Search}

%%%%%%%% Thesis Objectives %%%%%%%%
\chapter {Thesis Objectives}
\label{cha:Thesis Objectives}
1. Framwork for othello agents
2. Playground for RL algorithms, network optimizations, regularizations, etc.
3. High performance Othello Agent

%%%%%%%% Implementation %%%%%%%%
\chapter {Implementation}
\label{cha:Implementation}

%%%%%%%% Validation %%%%%%%%
\chapter {Validation}
\label{cha:Validation}

%%%%%%%% Conclusion %%%%%%%%
\chapter {Conclusion}
\label{cha:Conclusion}

%%%%%%%% Future Work %%%%%%%%
\chapter {Future Work}
\label{cha:Future Work}


%END Doc
%-------------------------------------------------------

\bibliographystyle{plain}
\bibliography{Thesis}

\end{document}
